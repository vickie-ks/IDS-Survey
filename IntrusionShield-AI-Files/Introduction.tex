\section{Introduction}
\label{Introduction}

% Recent advancements in network intrusion detection systems (NIDS) have incorporated various techniques, including machine learning and deep learning~\cite{Roshan2024, Chen2024, Meena2021, Mohammad2024, Ghadermazi2024, Aljuaid2024deep, Sadia2024, Islam2023, Jayalaxmi2022, Varshney2021, Dandaras2023, Rele2023, Amanoul2021, Alotaibi2023AML, macas2020review, Gutierrez2023enhancing, Vinayakumar2019deep, Wang2023robustness, Sowmya2023, Mirlekar2022}. These approaches have been applied to IoT, cloud, and traditional network infrastructures, utilizing novel frameworks and algorithms to improve detection rates and system robustness. Comprehensive reviews and research highlight the integration of anomaly-based models, data augmentation, and explainable AI for enhanced security.

The ever-increasing connectivity of digital systems and the ubiquity of Internet of Things (IoT) devices have profoundly transformed the cybersecurity landscape. While these advancements have unlocked unprecedented opportunities for innovation, they have also exposed networks to a rapidly growing array of sophisticated cyber threats. Intrusion Detection Systems (IDS) have emerged as essential components in the arsenal of modern cybersecurity, designed to detect and mitigate malicious activities targeting network and host systems~\cite{Mirlekar2022}. However, traditional IDS, which primarily rely on signature-based and rule-based techniques, often fail to address evolving threats such as zero-day attacks and adversarial intrusions~\cite{vinayakumar2019deep}. This limitation underscores the need for innovative approaches to safeguard critical infrastructures.

Recent advancements in Artificial Intelligence (AI), Machine Learning (ML), and Deep Learning (DL) have revolutionized the capabilities of IDS~\cite{Meena2021}. By leveraging the ability of ML algorithms to extract features from data and the hierarchical representation capabilities of DL models, AI-driven IDS can adapt to evolving attack patterns and effectively detect anomalies in large-scale networks~\cite{Islam2023}. For instance, convolutional neural networks (CNNs) and recurrent neural networks (RNNs) have demonstrated exceptional performance in classifying cyber threats~\cite{Varshney2021}, while hybrid models combining anomaly-based and signature-based detection offer enhanced robustness~\cite{Rele2023}.

This survey paper provides a comprehensive review of state-of-the-art research in IDS, focusing on the role of AI/ML/DL methodologies in enhancing intrusion detection. Specifically, it analyzes 20 key research contributions, categorizing them by their underlying techniques, datasets, and performance metrics. The evaluation metrics employed in these studies, including accuracy, precision, recall, F1 score, detection rate, false positive rate, and computational efficiency, are critically examined to highlight the strengths and limitations of each approach~\cite{Jayalaxmi2022}. These metrics not only serve as benchmarks for comparison but also guide researchers in addressing the challenges of building scalable, robust IDS frameworks.

Datasets play a crucial role in training and validating IDS models. Popular benchmarks like NSL-KDD, UNSW-NB15, and CICIDS2017 are extensively used to evaluate the efficacy of ML and DL techniques~\cite{Gutierrez2023}. However, the generalizability of these datasets to real-world scenarios remains a persistent challenge~\cite{Sadia2024}. In addition, adversarial machine learning poses significant threats to IDS, as attackers can exploit vulnerabilities in models to bypass detection~\cite{Alotaibi2023AML}. This survey addresses these challenges by emphasizing the importance of robust metrics and data diversity in IDS research.

The organization of this paper is as follows: Section~\ref{FoundationalConcepts} provides foundational concepts of IDS and the integration of AI/ML/DL techniques. Section~\ref{Methodology} outlines the methodology for selecting and categorizing the reviewed papers. Section~\ref{LiteratureReview} presents a detailed analysis of the surveyed studies, emphasizing the evaluation metrics and datasets used. Section~\ref{Discussion} discusses emerging trends, unresolved challenges, and future directions in IDS research. Finally, Section~\ref{Conclusion} concludes the paper by summarizing key findings and their implications for advancing IDS technologies.

By systematically analyzing the metrics, datasets, and methodologies employed in recent IDS research, this survey aims to bridge the gap between academic innovation and practical implementation. It serves as a comprehensive resource for researchers and practitioners, paving the way for the development of more resilient and adaptive intrusion detection systems.



% The rest of the paper is organized as follows. We provide a detailed
% description for our mobility model in Section~\ref{OurProtocol}.
% We evaluate the validity of the model in Section~\ref{SimulationStudy}, summarize the related work in Section~\ref{RelatedWork} and finally conclude in Section~\ref{Conclusion}.



