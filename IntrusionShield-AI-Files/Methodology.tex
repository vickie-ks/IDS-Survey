\section{Methodology}
\label{Methodology}

The methodology for this survey was meticulously designed to ensure a comprehensive and systematic review of Intrusion Detection Systems (IDS) research integrating Artificial Intelligence (AI), Machine Learning (ML), and Deep Learning (DL). This section outlines the criteria used for paper selection, the categorization framework applied, and the approach to data extraction and synthesis, providing a solid foundation for the insights presented in this survey.


\subsection{Paper Selection Criteria}
The selection of papers was guided by a clear and rigorous set of criteria to ensure relevance and quality. The primary focus was on studies that explored the application of AI/ML/DL in IDS, excluding traditional approaches that lacked these technological integrations. To maintain academic rigor, priority was given to papers published in reputable journals and conferences, including IEEE Access, Elsevier, and MDPI. The timeline for inclusion was also an important factor, with an emphasis on research published between 2019 and 2024 to capture the latest advancements in the field. Foundational papers that provided historical context and significant contributions were included regardless of their publication date.

Efforts were made to incorporate diverse perspectives by including studies that examined various IDS approaches, such as anomaly detection, signature-based systems, and hybrid methods. This diversity ensures a well-rounded understanding of the field. The initial search yielded over 50 papers, which were filtered based on these criteria, resulting in a curated selection of 20 high-impact studies for detailed analysis.




\subsection{Categorization of Reviewed Papers}
To facilitate a structured and focused review, the selected papers were categorized based on their primary contributions and areas of focus. The first category included papers that proposed or evaluated IDS architectures, covering host-based, network-based, and hybrid systems. This category provided insights into the design and operational principles of different IDS frameworks.

The second category focused on studies exploring AI/ML/DL methodologies. These papers detailed the use of models such as Support Vector Machines (SVMs), Convolutional Neural Networks (CNNs), and autoencoders, offering a comprehensive view of the techniques employed in IDS research. The third category included works centered on evaluation metrics and datasets, shedding light on how IDS performance is measured and the data used for training and testing. Lastly, the fourth category encompassed papers discussing challenges and future directions, highlighting unresolved issues such as adversarial threats, scalability, and dataset limitations, as well as proposing innovative solutions for advancing the field. This categorization provided a coherent framework for organizing and analyzing the diverse insights from the reviewed studies.




\subsection{Data Extraction and Analysis}
A systematic approach was employed to extract and analyze information from the selected papers. For each study, key details such as objectives, methodologies, evaluation metrics, datasets, and findings were documented. The analysis emphasized the unique contributions of each study while identifying common trends, recurring challenges, and gaps in the field.

The objectives and scope of the papers provided an understanding of the specific problems addressed and the context of the research. The methodologies offered insights into the AI/ML/DL techniques and IDS architectures utilized, highlighting innovative approaches and best practices. Evaluation metrics and datasets were critically assessed to gauge the robustness and generalizability of the findings. Finally, the strengths and limitations of each study were noted to provide a balanced perspective on their contributions and applicability.




\subsection{Ensuring Comprehensive Coverage}
To ensure comprehensive coverage, cross-referencing was performed among the selected papers to identify any missing studies or overlooked contributions. This iterative process helped refine the selection and ensured that the survey encompassed a wide range of perspectives and insights. By comparing and contrasting studies, unique contributions were highlighted, and redundancy was minimized, enhancing the overall coherence and depth of the review.




\subsection{Integration of Insights} The insights derived from the reviewed papers were synthesized into a cohesive narrative that forms the basis of this survey. The integration of these insights allowed for an in-depth exploration of IDS concepts, the advantages and challenges of integrating AI/ML/DL, and the evaluation of key performance metrics. This synthesis provided a holistic view of the field, offering valuable guidance for researchers and practitioners in cybersecurity.