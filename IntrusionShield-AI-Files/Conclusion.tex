\section{Conclusion}
\label{Conclusion}

The evolution of Intrusion Detection Systems (IDS) represents a critical response to the growing sophistication and scale of cyber threats. This paper has presented a comprehensive survey of recent advancements in IDS technologies, with a particular emphasis on the integration of Artificial Intelligence (AI), Machine Learning (ML), and Deep Learning (DL). By analyzing key research works and their methodologies, this study offers valuable insights into the current state, challenges, and future directions of IDS research.

One of the key findings of this survey is the transformative role of AI/ML/DL in enhancing IDS capabilities. These technologies have enabled the development of systems capable of detecting complex attack patterns, identifying anomalies in network traffic, and adapting to evolving cyber threats. As highlighted throughout the paper, techniques such as hybrid detection, Explainable AI (XAI), and adversarial training have addressed critical gaps in traditional IDS approaches. For example, XAI frameworks have improved the interpretability of black-box models, making AI-driven IDS more transparent and operationally viable~\cite{Islam2023}. Similarly, hybrid models have leveraged the strengths of both signature-based and anomaly-based detection, achieving higher accuracy and reduced false positive rates~\cite{Rele2023},~\cite{Mohammad2024}.

Despite these advancements, significant challenges persist. The lack of diverse and realistic datasets continues to hinder the generalizability and real-world applicability of IDS. While custom datasets introduced in studies like~\cite{Ghadermazi2024} and~\cite{aljuaid2024deep} have addressed specific use cases, the field still lacks standardized benchmarks that reflect the complexity of contemporary networks and attack vectors. Moreover, scalability, computational efficiency, and adversarial robustness remain critical areas requiring further innovation~\cite{Dandaras2023},~\cite{wang2023}.

This survey also underscores the importance of future research focusing on lightweight and energy-efficient architectures, particularly for IoT and edge computing environments. Techniques like model compression and edge-cloud collaboration offer promising avenues for achieving robust detection with minimal resource consumption~\cite{Gutierrez2023},\cite{Sowmya2023}. Additionally, advancements in real-time detection and distributed learning frameworks will be crucial in addressing the challenges posed by high-speed, large-scale networks\cite{Chen2024}.

The implications of these findings extend beyond technical improvements. By addressing challenges like interpretability, scalability, and adversarial threats, IDS can become more accessible to a broader range of applications, from enterprise networks to IoT ecosystems. This progress will not only enhance the resilience of modern infrastructures but also foster greater trust and adoption of AI-driven cybersecurity solutions.

In conclusion, the integration of AI/ML/DL has redefined the potential of IDS, transforming them into intelligent, adaptive, and robust systems capable of addressing the complexities of modern cyber threats. While significant challenges remain, the advancements and future directions outlined in this paper provide a roadmap for continued innovation in IDS research. By building on these foundations, researchers and practitioners can develop IDS technologies that are not only effective but also scalable, efficient, and resilient, ensuring a safer digital landscape for years to come.